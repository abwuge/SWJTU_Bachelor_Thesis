\swjtuChapter{示例}

\section{测试一下吧}

\subsection{测试一下公式}

无编号的公式:\(\int_a^b f(x)\mathrm{d}x=F(b)-F(a)\)

有编号的公式:
\begin{equation}
    \int_a^b f(x)\mathrm{d}x=F(b)-F(a)
    \label{eq:example}
\end{equation}

\swjtuExplanation{
    \(a\), 积分下界;
    \(b\), 积分上界;
    \(F(x)\), {原函数,如果需要打英文逗号(,),或者需要打英文分号(;),记得包围在大括号里};
}

来点正文,稍微长一点点吧。

\subsection{测试一下图表}

\begin{figure}[htbp]
    \centering
    \includegraphics[width=0.5\textwidth]{signatures/author.png}
    \caption{懒得画图了,随便放个签名}
    \label{fig:example}
\end{figure}

\begin{table}[htbp]
    \centering
    \caption{测试表格}
    \begin{tabularx}{0.8\linewidth}{|L|C|R|}
        \hline
        列1  & 列2  & 列3  \\
        \hline
        数据1 & 数据2 & 数据3 \\
        \hline
    \end{tabularx}
    \label{tab:example}
\end{table}

\begin{table}[htbp]
    \centering
    \caption{三线表}
    \begin{tabularx}{0.8\linewidth}{LCR}
        \toprule
        列1   & 列2   & 列3   \\
        \midrule
        数据11 & 数据12 & 数据13 \\
        数据21 & 数据22 & 数据23 \\
        \bottomrule
    \end{tabularx}
    \label{tab:example2}
\end{table}

\begin{table}[htbp]
    \centering
    \caption{主表格标题}
    \begin{subtable}[t]{0.45\textwidth}
        \centering
        \caption{子表A标题}
        \label{tab:subtableA}
        \begin{tabular}{|c|c|}
            \hline
            列1 & 列2 \\ \hline
            数据1 & 数据2 \\ \hline
        \end{tabular}
    \end{subtable}
    \hfill
    \begin{subtable}[t]{0.45\textwidth}
        \centering
        \caption{子表B标题}
        \label{tab:subtableB}
        \begin{tabular}{|c|c|}
            \hline
            列1 & 列2 \\ \hline
            数据3 & 数据4 \\ \hline
        \end{tabular}
    \end{subtable}
\end{table}

伪代码\ref{alg:flow-gta}或者\cref{alg:flow-gta}或者\autoref{alg:flow-gta}:

\begin{algorithm}[htbp]
\caption{测试算法}
\label{alg:flow-gta}
\begin{algorithmic}[1]
\Input A series of domains $\mathcal{A}$.
\Output $\frac{1}{N}$ 
\STATE Initialize
\FOR{$l = 0$ \textbf{to} $n-1$}
    \STATE $r \gets l + 1$
\ENDFOR
\end{algorithmic}
\end{algorithm}

\subsection{测试一下引用}

如\cref{eq:example}所示,测试一下公式引用。

如\cref{fig:example}所示,测试一下图片引用。

如\cref{tab:example}所示,测试一下表格引用。

如\cref{tab:example,tab:example2,tab:subtableA,tab:subtableB}所示,测试一下多个表格引用。

如\crefrange{tab:example}{tab:example2}所示,测试一下表格范围引用。

还有参考文件引用:
引用样式1\cite{example2025}。
引用样式2\parencite{example2025}。

\subsection{测试一下子节}
\subsubsection{测试一下子子节}

\begin{description}
    \item[介绍一个东西] 这里是具体内容描述文本,他需要有一点长。具体需要多长呢,大概需要比一页多一点。这样他就可以换行了。
    \item[介绍另一个] 另一段内容...
\end{description}
